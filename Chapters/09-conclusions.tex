\myChapter{Conclusions}\label{chap:conclusions} %and future work, �no?
\begin{flushright}{\slshape
    I didn't jump. I took a tiny step, \\and there conclusions were.} \\ \medskip
    --- {Buffy Summers. Phases. Buffy: the Vampire Slayer}
\end{flushright}
\minitoc\mtcskip
\vfill

% Tienes que reescribirlo pr�cticamente completo. Una tesis es "queremos probar esto". Las conclusiones son "hemos probado esto". Adem�s, las conclusiones normalmente se leen, "con la venia del tribunal", en voz alta y tienen que enfatizar lo que se ha probado y en qu� se ha avanzado el estado del arte, as� como qu� conclusiones se han deducido de los resultados obtendios - JJ



\section{Outlook}

% Yo no pondr�a subapartados. Y el trabajo futuro tiene que estar encaminado a probar alguna hip�tesis que haya surgido del trabajo hecho en la tesis, no a demostrar que no se ha hecho todo el trabajo. 

\subsection{Service Oriented EAs}
A web portal to centralize  services and implementations being offered to the community will be created as a future task. Also, interviews with EA practitioners with different skills in programming and different research areas will be performed, to validate if this change of paradigm is contributing to enhance their work.

% �Una entrevista? �En serio? �Eres de letras? Como lean esto se descojonan - JJ

\subsection{Parameter Adaptation in heterogeneous machines}

In the future it is planned to check the scalability of the approach presented in this thesis, using more computational nodes and larger problem instances. In addition, other parameters such as migration rate or crossover probability could be adapted to the execution nodes. Different implementation of these services could be automatically enabled depending of the current state of the EA and the node. Other appropriate benchmark services to analyse the algorithm will also be  used to  enable automatic parameter adaptation at runtime (online), with different nodes entering or exiting in the topology, or adapting the parameters to the current load of the system. 

% �No has probado la escalabilidad de esto? �En serio? Como lean esto los que no se hayan descojonado del primer trabajo futuro se descojonan de este. Adem�s, enseguda vas a tener la pregunta: �Es escalable? -JJ




%Todo esto esto por borrarlo tambi�n. El trabajo futuro tiene que ser espec�fico a la tesis, y tienes que poner algo que haya surgido de la tesis, no lo que no te haya dado tiempo de hacer. Adem�s, tienen que ser cosas que no se pudan hacer de otra forma.

\subsection{Evolutionary Art}

The future work for this research also includes performing more experiments with other kind of individuals, apart from circles: using other primitives, such as rectangles or triangles, for example. The use of textures and gradients will generate images with higher number of colors, obtaining more fidelity (more than 25\%) with the test image. Other metrics explained in Chapter \ref{chap:art} will also be  implemented. Finally, our intention is not only to create only static images, but use the {\em Processing} libraries to create evolutionary interactive art combining sounds and motion. 

More complex measurements will also be studied in future, taking into
account that the HSV is the colour mode that provides more information
during the evolution, having less noisy behaviour.
% �Es espec�fico de Osgiliath? No. Pues fuera - JJ

\subsection{RTS games}

In future work, other rules will be added to our algorithm (for
example, the ones that analyse the map, as the Exp-Genebot does) and
different enemies will be used. Other games used in the area of
computational intelligence in videogames, such as
Unreal\texttrademark~ or Super Mario\texttrademark~ will be tested.

% �Es espec�fico de Osgiliath? No. Pues fuera - JJ

\section{Publications related with this thesis}

\begin{itemize}
\item \person{Pablo Garc�a-S�nchez, J. Gonz�lez, Pedro A. Castillo, Maribel Garc�a Arenas, Juan Juli�n Merelo Guerv�s} \emph{Service oriented evolutionary algorithms}.  Soft Comput. 17(6): 1059--1075 (2013).


\item \person{Pablo Garc�a-S�nchez, J. Gonz�lez, Pedro A. Castillo, Juan Juli�n Merelo Guerv�s, Antonio Miguel Mora, Juan Lu�s Jim�nez Laredo, Maribel Garc�a Arenas} \emph{ A Distributed Service Oriented Framework for Metaheuristics Using a Public Standard}. In Proceeding of Nature Inspired Cooperative Strategies for Optimization. Studies in Computational Intelligence. Springer, 2010. p: 211--222.

\item \person{P. Garc�a-S�nchez, A. Fern�ndez-Ares, A. M. Mora, P. A. Castillo, J. Gonz�lez and J.J. Merelo} \emph{Tree depth influence in Genetic Programming for generation of competitive agents for RTS games}. Applications of Evolutionary Computation, EvoApplicatons 2010: EvoCOMPLEX, EvoGAMES, EvoIASP, EvoINTELLIGENCE, EvoNUM, and EvoSTOC, Proceedings. Springer, 2014 Lecture Notes in Computer Science (to appear).

% eliminado los trabajos o que no est�n a la altura o que no tienen mucho que ver con la tesis - JJ

\end{itemize}

During the development of this thesis, other works related with some of the topics of this thesis have also been published: Service Oriented Architecture \cite{GarciaSanchez2013Gateway,Garcia09UMM}, Cloud Computing \cite{meri2013cloud}, and other EA applications, such as Evolutionary Robotics \cite{Garcia2012testing}, video-games bot optimization \cite{Fernandez20111optimizing,Mora2012Genebot,FernandezAres2012adaptive,Esparcia10FPS}, inventory and route management \cite{Esparcia2009EVITA}, document transformation \cite{Garcia2008XSLT}, or heterogeneous hardware environments \cite{Garcia2009Mobile}.
