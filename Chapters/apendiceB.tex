\myChapter{Ap�ndice}\label{chap:apendiceB}
\minitoc\mtcskip
\vfill
\section{Demostraci�n del rango de funcionamiento de la\\Multiplicaci�n--\textsc{LIP}}\label{sec:demoMultLIP}
\lettrine{E}{n} este ap�ndice se demuestra que el rango de los resultados del operador \mbox{Multiplicaci�n--\textsc{LIP}} es $(0,M)$. Es decir, se desea demostrar:
\begin{equation}
0 < \alpha \LIPtimes \widehat{f} < M \qquad \forall \alpha \in \mathbb{R}, \forall \widehat{f} \in (0,M) \subseteq \mathbb{R}\label{eq:demostracionMultLIP}
\end{equation}

\noindent Se reescribe \eqref{eq:demostracionMultLIP} a partir de la definici�n matem�tica del operador \mbox{Multiplicaci�n--\textsc{LIP}} (definido en \eqref{eq:LIPtimes}), utilizando �nicamente operadores tradicionales.

\begin{equation}
0 < M - M \cdot \left(1-\frac{\widehat{f}}{M}\right)^\alpha < M \label{eq:demostracionMultLIPestandar}
\end{equation}

\noindent Esta demostraci�n se divide en dos trozos. En el primero, se demuestra que $0 < \alpha \LIPtimes \widehat{f}$; mientras que en una segunda parte se demuestra que $\alpha \LIPtimes \widehat{f} < M$.
\vfill
\clearpage

\subsection{Valores superiores a cero}

\noindent Se toma la primera parte que se desea demostrar $\left( 0 < \alpha \LIPtimes \widehat{f}\right)$.  Para confirmar dicha afirmaci�n, se utiliza el mecanismo de \emph{reducci�n al absurdo}. Para lo cual, se niega la afirmaci�n que se desea demostrar:

\begin{equation}
0 \not< M - M \cdot \left(1-\frac{\widehat{f}}{M}\right)^\alpha &\Rightarrow&
0 \ge M - M \cdot \left(1-\frac{\widehat{f}}{M}\right)^\alpha  \label{eq:demostracionMultLIPcero}
\end{equation}

\noindent Partiendo de \eqref{eq:demostracionMultLIPcero}, se opera hasta obtener inecuaci�n simple que permita un an�lisis directo. 

\begin{eqnarray}
0 \ge M \cdot \left(1 - \left(\frac{M-\widehat{f}}{M}\right)^\alpha \right) &\Rightarrow& 0 \ge 1 - \left(\frac{M-\widehat{f}}{M}\right)^\alpha \nonumber \\
\left(\frac{M-\widehat{f}}{M}\right)^\alpha \ge 1 &\Rightarrow & \frac{\left(M-\widehat{f}\right)^\alpha}{M^\alpha} \ge 1 \nonumber \\
\left(M-\widehat{f}\right)^\alpha &\ge& M^\alpha \label{eq:multLIPuno}
\end{eqnarray}

\noindent Se aplica el logaritmo neperiano a ambos partes de la inecuaci�n \eqref{eq:multLIPuno}.

\begin{eqnarray}
\log\left(M-\widehat{f}\right)^\alpha \ge \log\left(M^\alpha\right) &\Rightarrow& \alpha \cdot \log\left(M-\widehat{f}\right) \ge \alpha \cdot \log M \nonumber \\
\log\left(M-\widehat{f}\right) \ge \log M &\Rightarrow & \frac{\log\left(M - \widehat{f}\right)}{\log M} \ge 0 \nonumber \\
\log \left(M - \widehat{f} - M\right) \ge 0 &\Rightarrow & \log \left(-\widehat{f}\right) \ge 0 \label{eq:multLIPdos}
\end{eqnarray}

\noindent Sabiendo que, por definici�n, $\widehat{f} \in (0,M)\subseteq \mathbb{R}$, la inecuaci�n \eqref{eq:multLIPdos} no puede satisfacerse, puesto que no existe el logaritmo de un n�mero negativo. Consecuentemente, se puede afirmar con total seguridad que la hip�tesis de partida es falsa, y por tanto:
\begin{equation}
0 < \alpha \LIPtimes \widehat{f} \qquad \forall \alpha \in \mathbb{R}, \forall \widehat{f} \in (0,M) \subseteq \mathbb{R}\label{eq:multLIPrangoInferior}
\end{equation}

\noindent N�tese que no influye el valor de $\alpha$ en la determinaci�n del rango de salida.

\subsection{Valores inferiores a M}

\noindent La segunda parte de \eqref{eq:demostracionMultLIPestandar} pretende demostrar que $\left(\alpha \LIPtimes \widehat{f} < M\right)$.  Para dicha demostraci�n, como en el caso anterior, se utiliza el mecanismo de \emph{reducci�n al absurdo}. Se niega la afirmaci�n que se desea demostrar y se opera:

\begin{equation}
M - M \cdot\left(1 - \frac{\widehat{f}}{M}\right)^\alpha \not<  M\Rightarrow
M - M \cdot\left(1 - \frac{\widehat{f}}{M}\right)^\alpha \ge M \label{eq:demostracionMultLIPuno}
\end{equation}

\noindent Se sigue operando, obteniendo:

\begin{eqnarray}
M \cdot \left( 1 - \left(1 - \frac{\widehat{f}}{M}\right)^\alpha \right) \ge M & \Rightarrow & 1- \left(1 - \frac{\widehat{f}}{M}\right)^\alpha \ge 1 \nonumber \\
- \left(1 - \frac{\widehat{f}}{M}\right)^\alpha \ge 0 &\Rightarrow& \left(1 - \frac{\widehat{f}}{M}\right)^\alpha \le 0 \nonumber\\
\left(M - \widehat{f}\right)^\alpha \le 0 &\Rightarrow& M\cdot \widehat{f} \le 0 \quad \Rightarrow \quad M \le \widehat{f} \label{eq:multLIPtres}
\end{eqnarray}

\noindent Sin embargo, por definici�n, $\widehat{f} \in (0,M)\subseteq \mathbb{R}$. Por tanto, se tiene que $\widehat{f} < M$, lo que hace que la inecuaci�n \eqref{eq:multLIPtres} no sea cierta. Esto permite deducir que el punto de partida es falso, y se puede afirmar lo contrario:
\begin{equation}
\alpha \LIPtimes \widehat{f} < M \qquad \forall \alpha \in \mathbb{R}, \forall \widehat{f} \in (0,M) \subseteq \mathbb{R}\label{eq:multLIPrangoSuperior}
\end{equation}

\subsection{Rango completo}
\noindent Al unificar \eqref{eq:multLIPrangoInferior} y \eqref{eq:multLIPrangoSuperior} queda demostrado que el rango de salida del operador \mbox{Multiplicaci�n--\textsc{LIP}} es $(0,M)$ para cualquier valor del operando $\alpha \in \mathbb{R}, \widehat{f} \in (0,M) \subseteq \mathbb{R}$, tal y como se ha expuesto en \eqref{eq:demostracionMultLIP}.