\myChapter{Ap�ndice}\label{chap:apendiceA}
\minitoc\mtcskip
\vfill
\section{Demostraci�n del rango de funcionamiento de la\\ Suma--\textsc{LIP}}\label{sec:demoSumaLIP}
\lettrine{A}{\relax} continuaci�n, se demuestra que la aplicaci�n del operador \mbox{Suma--\textsc{LIP}} proporciona resultados en el mismo rango $(0,M)$ que el de los operandos de entrada. Es decir, se desea demostrar:
\begin{equation}
0 < \widehat{f} \LIPplus \widehat{g} < M \qquad \forall \widehat{f},\widehat{g} \in (0,M) \subseteq \mathbb{R}\label{eq:demostracionSumaLIP}
\end{equation}

\noindent Tomando la definici�n matem�tica del operador \mbox{Suma--\textsc{LIP}} (definido en \eqref{eq:LIPplus}), se puede reescribir \eqref{eq:demostracionSumaLIP} utilizando operadores tradicionales.

\begin{equation}
0 < \widehat{f} + \widehat{g} - \frac{\widehat{f}\cdot\widehat{g}}{M} < M \label{eq:demostracionSumaLIPestandar}
\end{equation}

\noindent Esta demostraci�n se va a dividir en dos partes. En la primera parte se va a demostrar que $0 < \widehat{f} \LIPplus \widehat{g}$; mientras que la segunda parte demostrar� que $\widehat{f} \LIPplus \widehat{g} < M$.
\vfill
\clearpage
\subsection{Valores superiores a cero}

\noindent Se toma la primera parte que se desea demostrar $\left( 0 < \widehat{f} \LIPplus \widehat{g}\right)$.  Para confirmar dicha afirmaci�n, se utiliza el mecanismo de \emph{reducci�n al absurdo}. Para lo cual, se niega la afirmaci�n que se desea demostrar y se opera:

\begin{equation}
0 \not< \widehat{f} + \widehat{g} - \frac{\widehat{f}\cdot\widehat{g}}{M} \Rightarrow
0 \ge \widehat{f} + \widehat{g} - \frac{\widehat{f}\cdot\widehat{g}}{M} \label{eq:demostracionSumaLIPcero}
\end{equation}

\noindent Para simplificar las operaciones, se va a utilizar el mismo operando. Esta simplificaci�n no significa, en ning�n caso, una falta de generalidad en esta demostraci�n, ya que puede ampliarse a dos operandos cualesquiera. Se reformula \eqref{eq:demostracionSumaLIPcero} utilizando un mismo operando, $\widehat{f}$, y se contin�a operando:\\

\begin{eqnarray}
0 \ge \widehat{f} + \widehat{f} &-& \frac{\widehat{f}\cdot\widehat{f}}{M} \nonumber \\
0 \ge 2\cdot\widehat{f} - \frac{\widehat{f}^2}{M} &\Rightarrow& 0 \ge 2\cdot M\cdot\widehat{f} - \widehat{f}^2 \nonumber \\
\frac{\widehat{f}^2}{M} \ge 2\cdot \widehat{f} &\Rightarrow& \frac{\widehat{f}}{2} \ge M \label{eq:sumaLIPuno}
\end{eqnarray}

\noindent Sabiendo que por definici�n, $\widehat{f} \in (0,M)\subseteq \mathbb{R}$, a continuaci�n, se calculan los l�mites de los valores extremos del rango de $\widehat{f}$ para \eqref{eq:sumaLIPuno}. 

\begin{eqnarray}
\lim_{\widehat{f} \to 0} \left(\frac{\widehat{f}}{2}\right) &=& 0 \label{eq:sumaLIPlimiteA} \\
\lim_{\widehat{f} \to M} \left(\frac{\widehat{f}}{2}\right) &=& \frac{M}{2} \label{eq:sumaLIPlimiteB}
\end{eqnarray}
\noindent Para \eqref{eq:sumaLIPlimiteA}, el an�lisis es trivial, ya que el �nico caso para el que $0 \ge M$, se produce cuando $M=0$. Esto representar�a un rango de trabajo nulo, aspecto que no tiene sentido tanto desde el punto de vista matem�tico como desde el punto de vista pr�ctico.\\
\noindent La demostraci�n para \eqref{eq:sumaLIPlimiteB} tambi�n es trivial, puesto que $\frac{M}{2} \ge M$ es v�lido �nicamente si $M=0$, que como se ha indicado anteriormente, no es matem�ticamente posible.\\
\noindent Con lo cual, se puede afirmar con total seguridad que la hip�tesis de partida es falsa, y por tanto:
\begin{equation}
0 < \widehat{f} \LIPplus \widehat{g} \label{eq:sumaLIPrangoInferior}
\end{equation}

\subsection{Valores inferiores a M}

\noindent La segunda parte de \eqref{eq:demostracionSumaLIPestandar} pretende demostrar que $\left(\widehat{f} \LIPplus \widehat{g}< M\right)$.  Para dicha demostraci�n, como en el caso anterior, se utiliza el mecanismo de \emph{reducci�n al absurdo}. Se niega la afirmaci�n que se desea demostrar y se opera:

\begin{equation}
\widehat{f} + \widehat{g} - \frac{\widehat{f}\cdot\widehat{g}}{M} \not<  M\Rightarrow
\widehat{f} + \widehat{g} - \frac{\widehat{f}\cdot\widehat{g}}{M} \ge M \label{eq:demostracionSumaLIPuno}
\end{equation}

\noindent Como en el caso anterior, para simplificar las operaciones, se realizan los c�lculos utilizando el mismo operando. Por lo que, reescribiendo \eqref{eq:demostracionSumaLIPuno} con un mismo operando, $\widehat{f}$, se obtiene:\\

\begin{eqnarray}
\widehat{f} + \widehat{f} &-& \frac{\widehat{f}\cdot\widehat{f}}{M} \ge M \nonumber \\
2\cdot\widehat{f} - \frac{\widehat{f}^2}{M} \ge M &\Rightarrow& 2\cdot M\cdot\widehat{f} - \widehat{f}^2 \ge M^2 \nonumber \\
0 \ge M^2 - 2\cdot M\cdot\widehat{f} + \widehat{f}^2 &\Rightarrow& 0 \ge \left(M - \widehat{f}\right)^2 \nonumber \\
0 \ge M - \widehat{f} &\Rightarrow& \widehat{f} \ge M\label{eq:sumaLIPtres}
\end{eqnarray}

\noindent Sin embargo, por definici�n, $\widehat{f} \in (0,M)\subseteq \mathbb{R}$. Por tanto, se tiene que $\widehat{f} < M$, lo que hace que la inecuaci�n \eqref{eq:sumaLIPtres} no sea v�lida en ning�n caso. Esto permite deducir que el punto de partida es falso, y se puede afirmar lo contrario:
\begin{equation}
\widehat{f} \LIPplus \widehat{g} < M \label{eq:sumaLIPrangoSuperior}
\end{equation}

\subsection{Rango completo}
\noindent Al unificar \eqref{eq:sumaLIPrangoInferior} y \eqref{eq:sumaLIPrangoSuperior} queda demostrado que el rango de salida del operador \mbox{Suma--\textsc{LIP}} es $(0,M)$ para cualesquiera dos operandos $\widehat{f}, \widehat{g} \in (0,M) \subseteq \mathbb{R}$, tal y como se ha expuesto en \eqref{eq:demostracionSumaLIP}.