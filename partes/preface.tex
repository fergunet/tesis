%%%%%%%%%%%%%%%%%%%%%%%%%%%%%%%%%%%%%%%%%%%%%%%%%%%%%%%%%%%%%%%%%%%%%%%%%%%%%%%
%%                                                                           %%
%%                                  Prefacio                                 %%
%%                                                                           %%
%%%%%%%%%%%%%%%%%%%%%%%%%%%%%%%%%%%%%%%%%%%%%%%%%%%%%%%%%%%%%%%%%%%%%%%%%%%%%%%


\chapter* {Preface}


Evolutionary Algorithms are a set of population based stochastic search techniques able to solve optimisation problems in reasonable time. However, the execution times of EAs can be high for very demanding problems and parallelism arises as an alternative to improve the algorithm performance and to speed up times to solutions. 

In that context, this thesis presents a spatially structured EA able to take full advantage of the large amount of available resources in P2P platforms. Such an approach defines a decentralised population structure by means of a P2P protocol in which every individual has a limited number of neighbours with the mating choice locally restricted within the P2P neighbourhood. The emergent population structure behaves as a small-world topology and plays an important role in the preservation of the genetic diversity. That way, population sizes can be minimised and  execution times improve.



Nevertheless, there are remaining challenges towards an efficient design of P2P EAs. Questions such as \emph{decentralisation} (such a computation paradigm is devoid of any central server), \emph{scalability} (since P2P systems  are large-scale networks) or \emph{fault tolerance} (given that computational resources are added and eliminated dynamically) become of the maximum interest and have to be addressed. Therefore, this thesis focuses on analysing such issues (i.e. decentralisation, scalability and fault-tolerance) in order to conclude the viability of the Peer-to-Peer Evolutionary Computation paradigm.








%The aim of this thesis is to dive on the
%issues, properties and performance of Peer-to-Peer (P2P) Evolutionary
%Algorithms (EAs). As in any other distributed EA, a P2P EA faces % has
				% got, no faces - JJ
%two aspects,
% the algorithmic and the computational performance. The first is
% related to the structural adaptations that the algorithm suffers when
% No son adaptaciones estructurales, sino cambios del algoritmo hasta el
 % punto que es un algoritmo diferente - JJ
% deployed on several loosely coupled processors while the latter corresponds to the
% computational speedup that can be expected. In fact, distributed EAs
% are studied  as a way of preserving genetic diversity while improving
% the runtime  of the algorithm \cite{cantu:parallelga}. To this end, a
% good understanding on the physical platform, P2P systems in this case,
% can be leveraged in its  design. 

%P2P systems are defined by application level % No uses ``are'' ni otros
				% verbos copulativos. Está compuesto,
				% consiste en, tiene...
%networks (ALN) able to constitute a single virtual system composed
%of a potentially large number of interconnected resources. 

% Esto de aquí no sé a santo de qué viene
%The
%computational power is provided by a group of users connected to the Internet who share their spare CPU cycles
%(e.g. the BOINC project is a successful case of virtual supercomputer based on 
%volunteers sharing their computers' CPU cycles \cite{seti}). 
% Si viene a cuento, necesitas explicarlo mejor - JJ

%However,
% there are still many challenging issues in the parallelization of  EAs in P2P systems. Questions such as
% \emph{decentralization} (such a computation paradigm is devoid of any
% central server), \emph{scalability} (since P2P systems  are
% large-scale networks) or \emph{fault tolerance} (given that resources
% are added and eliminated dynamically, often as a consequence of a
% decision from an user that volunteers CPUs under his control) become
% of the maximum interest and have to be addressed. 
% ... which is something we will do here? no puedes sacar la espada sin
 % envainarla ensangrentada - JJ
 
%On the other hand, P2P systems define a rich set of topologies % for the
% interconnection of nodes at application level, so-called overlay
% networks \cite{wehrle05:p2p}. 
% Overlay networks son lo mismo que ALN, no? Si estás hablando de la
% topología específicamente, habla de eso, no uses el término general -
% JJ 
%
% Esto tampoco sé a santo de qué viene... 
%In parallel, within the EC area, spatially structured EAs focus on the
%study of different topologies as population structure for an EA (see
%\cite{tomassini} for a survey). Hence, a distributed P2P EA can be
%designed as a spatially structured EA in which the population structure
%is defined by a P2P overlay network. 
% como todo, introdúcelo adecuadamente en la argumentación - JJ

% Introducir standard approaches.... para ligar con el párrafo siguiente

%Additionally to the standard approaches in EAs stating that the mate choice depends just on fitness (\emph{panmixia}), spatially structured EAs  define a population in which any given individual has a restricted number of neighbours and the chances for mating are, therefore, reduced within the neighbourhood. Population structures can be modeled as a graph in which the vertices are individuals\footnote{In this chapter, we refer equally to the terms individual and node, since each individual has its own schedule and could potentially be placed in a different node} and edges represent relationships between them. Since a graph can be easily mapped to a network topology, a spatially structured EA can be easily distributed.


%The key to the P2P EA presented in this thesis is the Evolvable Agent
%model (\emph{EvAg}) proposed by the author and coworkers in % Ten
				% cuidado con estos copy/paste de los
				% papers 
%\cite{laredo:cec2008}. It consists of a fine grained and decentralized
%approach for parallelizing EAs in which there is a population of
%concurrent and self-scheduled agents performing the evolutionary steps
%of selection, variation and evaluation of individuals. The population
%structure is based on the gossiping P2P protocol called {\em 
%  newscast} and presented in \cite{jelasity:newscast}. {\em Newscast} builds a small-world topology 
%in which every pair of nodes are connected through
%a short sequence of intermediate nodes \cite{wattsstrogatz}.
% Tienes que dejar claro de qué es de lo que vas a hablar en la tesis y
% cuál es tu aportación. Aquí estás mezclando lo tuyo con el newscast y
% con varias cosas más - JJ

% Ahora voy a hablar de los issues....

% Esto deberías moverlo a la parte donde hablas de P2P por primera vez -
% JJ 
%An inherent advantage of using P2P protocols as population structure
%is that they are designed to tackle large-scale graphs and they
%present consequently a good scalability behavior 
% La escalabilidad va a depender siempre del algoritmo, no de la red
% subyacente (aunque ayuda). Un algoritmo en un sistema centralizado
% puede tener una escalabilidad mejor que la que se puede conseguir en
% una red P2P. Particulariza a un problema o conjunto de problemas
% determinado, o formúlalo de otra forma. Habla siempre con propiedad y
% no hagas afirmaciones discutibles que no estén relacionadas con el
% tema principal de la tesis. De hecho, algunos algoritmos (una simple
% búsqueda) puede escalar muy mal en un sistema totalmente
% descentralizado - JJ
%(see the study of
%scalability of the newscast protocol by Voulgaris et al. in
%\cite{spyros:robustscalable}). 
%This way, results in \cite{laredo08:large} about the scalability of the \emph{EvAg} model are specially remarkable under large instances of hard optimization problems. As a general property of optimization problems, the  evaluation cost scales with respect to the size of the problem instance. Hence, large instances imply a bigger computational cost on the evaluation function. Additionally, the problem complexity increases with size, making the problem more difficult to solve. Resolution methods based on population, as EAs, require large population sizes in order to tackle such instances with enough reliability and P2P systems are a large and mostly free source of computing power.


%Estás hablando de todo a la vez. ¿Qué es lo que buscas con el diseño de
%tu algoritmo? ¿Qué propiedas tiene de forma inherente, aunque no las
%buscaras específicamente? 
%Besides scalability, fault tolerance is also an important issue in the
%design of P2P applications since resources in P2P systems are prone to
%failure. Such systems are subject to the dynamics of peers: a node joins
%the system, contributes some resources and leaves it afterwards
%\cite{merelo08:agajaj}. The independent arrival and departure of
% eso está bien, que cites mi agajaj :-) - JJ
% Es estado del arte ;-) - Juanlu
%thousands of peers causes a collective effect called \emph{churn}. The
%\emph{EvAg} model has been tested  in \cite{laredo08:churn} under
%different \emph{churn} conditions. In spite of the departure of nodes,
%possibly containing valid solutions, the \emph{EvAg} model is able to
%reach the success criterion (a success rate of 0.98) in all the
%test-cases. Furthermore, assuming no restrictions in the amount of
%available peers, the runtime of the algorithm scales
%independently of either the \emph{churn} scenario and the population
%size, which confirms that the \emph{EvAg} model is robust and fault
%tolerant. 


%Finally, there are two major current lines of application in which P2P EAs are promising: Results in \cite{laredo08:large} show the suitability of such algorithms for tackling large instances of problems with high requirements in computing power via massive scalability in P2P systems. On the other hand, the \emph{EvAg} model has been applied to dynamic optimization problems (DOP) in \cite{laredo08:dop}. The P2P approach outperforms the results of the state of the art algorithm Self-Organizing Random Immigrants GA (SORIGA) presented in \cite{soriga}. Key to this is that, as spatially structured EAs, P2P EAs preserve genetic diversity by relaxing the environmental selection pressure at the small-world relationships between individuals.

\clearpage

%%%%%%%%%%%%%%%%%%%%%%%%%%%%%%%%%%%%%%%%%%%%%%%%%%%%%%%%%%%%%%%%%%%%%%%%%%%%%%%