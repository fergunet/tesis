%*******************************************************
% Abstract
%*******************************************************
%\renewcommand{\abstractname}{Abstract}
\pdfbookmark[1]{Resumen}{Resumen}
\begingroup
\let\clearpage\relax
\let\cleardoublepage\relax
\let\cleardoublepage\relax

\chapter*{Resumen}
Esta Tesis Doctoral trata de probar que usar un marco determinado es el m�s adecuado para solucionar problemas actuales en los Algoritmos Evolutivos en desarrollo, integraci�n, interoperabilidad y dinamismo. Para validar este objetivo la tesis se ha dividido en 3 partes.

Primero se analizan las nuevas tendencias en investigaci�n en Algoritmos Evolutivos para detectar sus principales carencias y se propone el uso de Arquitectura Orientada a Servicios (AOS) como el marco para resolver estas carencias.

La siguiente parte de esta tesis propone una metodolog�a (SOA-EA) para desarrollar Algoritmos Evolutivos Orientados a Servicios (AEOS). Esta metodolog�a se ha utilizado para implementar un framework para AEOS, llamado OSGiLiath, utilizando una tecnolog�a concreta (OSGi). Se presentan diferentes experimentos para validar cada una de las mejoras propuestas. 

OSGiLiath se ha usado tambi�n para desarrollar un m�todo para adaptar el tama�o de poblaci�n de un AE distribuido a la potencia de calculo de los nodos que forman el sistema distribuido. Finalmente, SOA-EA y OSGiLiath se han utilizado en otras aplicaciones: generaci�n de un bot inteligente para un juego en tiempo real y arte generativo.

%Esto es lo que deber�as haber escrito para empezar y no perderlo
%nunca de vista. �C�mo puedes escribir la tesis si no sabes lo que
%pretende? - JJ FERGU: estaba en el github, pero lo pongo aqu� (luego lo traduzco cuando tenga el visto bueno)

\vfill

\pdfbookmark[1]{Abstract}{Abstract}
\chapter*{Abstract}
This Ph. Thesis aims to ...


\endgroup

\vfill